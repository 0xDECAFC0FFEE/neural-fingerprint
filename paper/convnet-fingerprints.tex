\documentclass{article}
\usepackage{nips14submit_e,times}
\usepackage{hyperref}
\usepackage{natbib}
\ProvidesPackage{preamble}

\usepackage{url}
\usepackage{array}
\usepackage{amsmath,amssymb,amsfonts,textcomp}
\usepackage{booktabs}
\usepackage{relsize}
\usepackage{nicefrac}
\usepackage{graphicx}
\usepackage{rotating}
\usepackage{nth}
\usepackage{acronym}
\usepackage{bm}

\newcommand{\binarysum}{\sum_{\bf{x} \in \{0,1\}^D}}
\newcommand{\expect}{\mathbb{E}}
\newcommand{\expectargs}[2]{\mathbb{E}_{#1} \left[ {#2} \right]}
\newcommand{\var}{\mathbb{V}}
\newcommand{\varianceargs}[2]{\mathbb{V}_{#1} \left[ {#2} \right]}
\newcommand{\variance}{\mathbb{V}}
\newcommand{\cov}{\operatorname{cov}}
\newcommand{\Cov}{\operatorname{Cov}}
\newcommand{\covarianceargs}[2]{\Cov_{#1} \left[ {#2} \right]}
\newcommand{\colvec}[2]{\left[ \begin{array}{c} {#1} \\ {#2} \end{array} \right]}
\newcommand{\tbtmat}[4]{\left[ \begin{array}{cc} {#1} & {#2} \\ {#3} & {#4} \end{array} \right]}

%\newcommand{\covskinny}[2]{\var\!\left(#1\middle\vert#2\right)} 

\newcommand{\acro}[1]{\textsc{#1}}
%\newcommand{\vect}[1]{\boldsymbol{#1}}
\newcommand{\vect}[1]{{\bf{#1}}}
\newcommand{\mat}[1]{\mathbf{#1}}
\newcommand{\pderiv}[2]{\frac{\partial #1}{\partial #2}}
\newcommand{\npderiv}[2]{\nicefrac{\partial #1}{\partial #2}}

\newcommand{\pha}{^{\phantom{:}}}

\newcommand{\argmin}{\operatornamewithlimits{argmin}}
\newcommand{\argmax}{\operatornamewithlimits{argmax}}

% The following designed for probabilities with long arguments

\newcommand{\Prob}[2]{P\!\left(\,#1\;\middle\vert\;#2\,\right)}
\newcommand{\ProbF}[3]{P\!\left(\,#1\!=\!#2\;\middle\vert\;#3\,\right)}
\newcommand{\p}[2]{p\!\left(#1\middle\vert#2\right)}
\newcommand{\po}[1]{p\!\left(#1\right)}
\newcommand{\pF}[3]{p\!\left(\,#1\!=\!#2\;\middle\vert\;#3\,\right)} 
\newcommand{\mean}[2]{{m}\!\left(#1\middle\vert#2\right)}
%\newcommand{\novmean}[2]{{m}\!\left(#1\middle\vert#2\right)}
%\newcommand{\novcov}[2]{\var\!\left(#1\middle\vert#2\right)}
%\newcommand{\cov}[2]{\var\!\left(#1\middle\vert#2\right)} 
%\newcommand{\pskinny}[2]{p\!\left(#1\;\middle\vert\;#2\right)}
%\newcommand{\meanskinny}[2]{{m}\!\left(#1\middle\vert#2\right)}
%\newcommand{\covskinny}[2]{\var\!\left(#1\middle\vert#2\right)} 

\newcommand{\vI}{\mat{I}}
\newcommand{\vX}{\mat{X}}
\newcommand{\vY}{\mat{Y}}
\newcommand{\vZ}{\mat{Z}}
\newcommand{\vK}{\mat{K}}
\newcommand{\vs}{\vect{s}}
\newcommand{\va}{\vect{a}}
\newcommand{\vA}{\vect{A}}
\newcommand{\vb}{\vect{b}}
\newcommand{\vB}{\mat{B}}
\newcommand{\vR}{\mat{R}}
\newcommand{\vS}{\mat{S}}
\newcommand{\vu}{\vect{u}}
\newcommand{\vk}{\vect{k}}
\newcommand{\vc}{\vect{c}}
\newcommand{\vC}{\mat{C}}
\newcommand{\vw}{\vect{w}}
\newcommand{\vx}{\vect{x}}
\newcommand{\vy}{\vect{y}}
\newcommand{\vz}{\vect{z}}
\newcommand{\vmu}{\vect{\mu}}
\newcommand{\vpi}{\vect{\pi}}
\newcommand{\vphi}{\vect{\phi}}
\newcommand{\vSigma}{\mat{\Sigma}}
\newcommand{\vtheta}{\vect{\theta}}
\newcommand{\vl}{\vect{l}}
\newcommand{\vq}{\vect{q}}
\newcommand{\vf}{\vect{f}}
\newcommand{\vg}{\vect{g}}
\newcommand{\vell}{\vect{\ell}}
\newcommand{\ve}{\vect{\epsilon}}
\newcommand{\vzero}{\vect{0}}
\newcommand{\vone}{\vect{1}}

\newcommand{\He}{\mathcal{H}}
\newcommand{\normx}[2]{\left\|#1\right\|_{#2}}
\newcommand{\Hnorm}[1]{\normx{#1}{\He}}
\newcommand{\mmd}{{\rm MMD}}


\newcommand{\mf}{\bar{\vf}}

\newcommand{\st}{_\star}

\newcommand{\inv}{^{{\mathsmaller{-1}}}}
\newcommand{\tohalf}{^{{\mathsmaller{\nicefrac{1}{2}}}}}

\newcommand{\N}[3]{\mathcal{N}\!\left(#1|#2,#3\right)}
\newcommand{\bN}[3]{\mathcal{N}\big(#1|#2,#3\big)}
\newcommand{\boldN}[3]{\text{\textbf{\mathcal{N}}}\big(#1;#2,#3\big)}
\newcommand{\ones}[1]{\mat{1}_{#1}}
\newcommand{\eye}[1]{\mat{E}_{#1}}
\newcommand{\tra}{^\ensuremath{\mathsf{T}}}
\newcommand{\trace}{\operatorname{tr}}
\newcommand{\deq}{:=}
\newcommand{\degree}{^\circ}

\DeclareMathOperator{\chol}{chol}
\DeclareMathOperator{\diag}{diag}

\newcommand{\gp}{{\acro{gp}}}
\newcommand{\bmc}{{\acro{bmc}}}
\newcommand{\bq}{{\acro{bq}}}
\newcommand{\sbq}{{\acro{sbq}}}

\newenvironment{narrow}[2]{%
  \begin{list}{}{%
  \setlength{\topsep}{0pt}%
  \setlength{\leftmargin}{#1}%
  \setlength{\rightmargin}{#2}%
  \setlength{\listparindent}{\parindent}%
  \setlength{\itemindent}{\parindent}%
  \setlength{\parsep}{\parskip}}%
\item[]}{\end{list}}

\newtheorem{prop}{Proposition}
\newtheorem{cor}{Corollary}
\newtheorem{lem}{Lemma}


\def\ie{i.e.\ }
\def\eg{e.g.\ }
\def\iid{i.i.d.\ }
\def\simiid{\sim_{\mbox{\tiny iid}}}
\def\eqdist{\stackrel{\mbox{\tiny d}}{=}}

\def\Reals{\mathbb{R}}

\def\Uniform{\mbox{\rm Uniform}}
\def\Bernoulli{\mbox{\rm Bernoulli}}
\def\GP{\mathcal{GP}}

\def\inputVar{x}
\def\InputVar{X}
\def\InputSpace{\mathcal{X}}
\def\outputVar{y}
\def\OutputSpace{\mathcal{Y}}
\def\function{f}
\def\kernel{k}
\def\KernelMatrix{K}
\def\SumKernel{\sum}
\def\ProductKernel{\prod}
\def\expression{e}

\def\SE{\acro{SE}}
\def\Per{\acro{Per}}
\def\RQ{\acro{RQ}}
\def\Lin{\acro{Lin}}

\def\subexpr{{\cal S}}
\def\baseker{{\cal B}}
\def\numWinners{k}

\newcommand{\kSE}{{\acro{SE}}}
\newcommand{\kC}{{\acro{C}}}
\newcommand{\kPer}{{\acro{Per}}}
\newcommand{\kLin}{{\acro{Lin}}}
\newcommand{\kWN}{{\acro{WN}}}
\newcommand{\kCP}{{\acro{CP}}}
\newcommand{\kCW}{{\acro{CW}}}
\newcommand{\kRQ}{{\acro{RQ}}}

\usepackage[margin=1.25in]{geometry}

\title{Learning Molecular Fingerprints}


\author{
TBD
\And
TBD
\And
Ryan P. Adams\\
Harvard University
\And
}

% The \author macro works with any number of authors. There are two commands
% used to separate the names and addresses of multiple authors: \And and \AND.
%
% Using \And between authors leaves it to \LaTeX{} to determine where to break
% the lines. Using \AND forces a linebreak at that point. So, if \LaTeX{}
% puts 3 of 4 authors names on the first line, and the last on the second
% line, try using \AND instead of \And before the third author name.

\newcommand{\fix}{\marginpar{FIX}}
\newcommand{\new}{\marginpar{NEW}}

%\nipsfinalcopy % Uncomment for camera-ready version

\begin{document}


\maketitle

\begin{abstract}
Predicting the properties of new compounds requires a representation of molecules that allows generalization.
Currently, such representations are based on hand-coded features such as the Morgan circular fingerprints, which have [various limitations].
We introduce a set of fingerprints based on convolutional neural nets which generalize commonly-used fingerprints.
\end{abstract}



\section{Desirable Properties of Molecular Representations}

\begin{itemize}

\item invariant to global orientation
\item invariant to atom re-labeling
\item invariant to the SMILES representation (at the very least)
\item NOT invariant to reordering locally (maintain information about the orientation)
    example:  fragment1 - C - O - fragment2 is not the same as fragment1 - O - C - fragment2 
   \item speed
\end{itemize}

\subsection{Design decisions}

\paragraph{How much computation to perform at each layer?}
i.e. how complicated should we make the function that goes from one layer of the graph to the next?

\paragraph{Pyramidal versus block-shaped computational graph}

\paragraph{Building parse trees}
For the pyramidal architecture, we need to decide on a parse tree of the molecule.
One way to do this in a 'soft' way might be to max-pool over many different local parsings.
This could be done at multiple layers, which would limit the exponential blowup.

\paragraph{Preserve asymmetries explicitly or implicitly}
If we tie the weights of all neighboring vertices, then ordering information is lost locally, although it is still preserved implicitly in the relations between nodes in the next layer.

\paragraph{Should edges be nodes in the graph}
Not having them be nodes is slightly simpler, so we'll do that for now.

\paragraph{Do we directly tell each layer which depth it's at?}  Either through different weights, or features that indicate the layer of computation.

\paragraph{How many layers?}

\paragraph{What kind of max-pooling?}
k-max pooling might be appropriate.

\subsection{Things we might be able to predict}

\begin{itemize}
\item Molecule size
\item Molecule diameter (both graph diameter and physical diameter)
\item Solubility
\item Dipole moment of the entire molecule
\item homo/lumo
\item peak of emission or absorbtion spectra
\item affinity to particular targets (for drug design)
\end{itemize}


\section{Related work}

\paragraph{Fingerprints}

[Cite Morgan] and \citet{ECFP2010} developed extended circular fingerprints.
These fingerprints map identical molecules to the same set of fingerprints.
These same fingerprints have been pressed into service as a measure of similarity.

However, for a meaningful measure of similarity, 

\paragraph{Convolutional Neural Networks}

Convolutional neural networks have been used to model images, speech, and time series\citep{lecun1995convolutional}.
However, standard convolutional architectures use a fixed computational graph, making them difficult to apply to objects of varying size or structure, such as molecules.
More recently, \citet{KalchbrennerACL2014} developed a convolutional neural network architecture for modeling sentences of varying length and structure.

\paragraph{Recursive Neural Networks}

\citet{socher2011semi} and \citet{socher2011dynamic} use a pyramidal architecture for performing inference on variable-length sentences.

\paragraph{Neural nets for QSAR}

\cite{dahl2014multi} used standard deep neural networks, and didn't do any feature engineering.

\paragraph{Machine learning for identifying promising molecules}

\citet{Eckert2007225, bergeron2011modeling} provide reviews of the field.
\citet{tingley2014towards} used a variety of standard machine learning algorithms to predict the photovoltaic efficiency of organic molecules.

\section{Experiments}

[How much is conceptual purity worth?  We can try using only topology, or include lots of hand-engineered features that we think will be useful.]

\subsection{Things we'll compare against}

\begin{itemize}
\item Morgan fingerprints
\item Coulomb matrices (Cite)
\end{itemize}

\subsection{Datasets}

\subsection{Interpretability}

[Idea: Use nested dropout to allow a variable-sized descriptor.]
Explain that it's analogous to PCA for neural nets


[Wishlist: Include figures showing which fragments maximally activate different features - hopefully showing that they correspond to interpretable, familiar concepts]


\section{Conclusions}


\subsubsection*{Acknowledgements}

\bibliographystyle{plainnat}
\bibliography{references}


\end{document}
